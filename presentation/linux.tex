\documentclass[12pt,utf8,notheorems,compress]{beamer}

\usepackage[utf8]{inputenc}
\usepackage{default}

\usepackage{lmodern,listings}

% \usepackage{type1cm}
% \usepackage{type1ec}
% \RequirePackage{fix-cm}  %(Die Warnungen wurden schon durch lmodern beseitigt.)

% \usetheme{Berlin}
\usetheme{Warsaw}
\useoutertheme{split}
\usecolortheme{seahorse}
\usepackage{kurier}
\useinnertheme{rectangles}
\setbeamertemplate{navigation symbols}{}
\setbeamertemplate{footline}{}
\setbeamertemplate{headline}{}




% \usepackage[ngerman]{babel}
\usepackage[T1]{fontenc} % Trennung bei Woertern mit Umlauten
\usepackage{amssymb,amsmath,amsthm}
\usepackage{url,tikz}
\usepackage{graphicx}
\usepackage{comment,microtype}
\usepackage{listings}
\usepackage{pst-all}

%\usepackage{tikz-cd} % for commutative diagrams, but didn't work here
\usepackage{amsmath,amscd} % for (very simple) commutative diagrams

%\hyphenation{mani-fold}
%\hyphenation{sub-mani-fold}

\setlength\parskip{\medskipamount}
\setlength\parindent{0pt}

\newcommand{\slogan}[1]{%
  \begin{center}%
    \setlength{\fboxrule}{2pt}%
    \setlength{\fboxsep}{-3pt}%
    {\usebeamercolor[fg]{item}\fbox{\usebeamercolor[fg]{normal
    text}\parbox{0.9\textwidth}{\begin{center}#1\end{center}}}}%
  \end{center}%
}

% top color=#2!10, bottom color=#2!90

\newcommand{\mybox}[2]{%
         \begin{center}%
            \begin{tikzpicture}%
                \node[rectangle, draw=#2, rounded corners=5pt, inner xsep=5pt, inner ysep=6pt, outer ysep=10pt]{
                \begin{minipage}{0.75\linewidth}#1\end{minipage}};%
            \end{tikzpicture}%
         \end{center}%
}

\everymath{\displaystyle}

\title{Hacking Workshop -- Mathecamp 2016 in Windischleuba}

\author{Sven Prüfer}

%\institute{Institut f\"ur Mathematik, Universit\"at Augsburg}

\date{\today}

%\titlegraphic{
% \includegraphics[width=8cm]{./Uni_Aug_Logo_MNF_CMYK.pdf}
%}

\setlength{\unitlength}{1cm}

\begin{document}

\setbeameroption{show notes}
\setbeamertemplate{note page}[plain]

\begin{frame}
 \titlepage
\end{frame}

\begin{frame}[shrink=25]
%\frametitle{Inhalt}
\vspace{1cm}
\tableofcontents
\end{frame}

\section{Hinweise}

\begin{frame}
  \frametitle{Hinweise}
  \begin{center}
    \Huge Hinweise
  \end{center}
\end{frame}

  \begin{frame}
    \frametitle{Rechtliches}
    Macht niemals irgendsoetwas auf Rechnern, auf denen ihr das nicht dürft oder von deren Betreibern ihr kein Einverständnis habt.
    \pause \vfill
    Und auf gar keinen Fall in der Schule!
  \end{frame}

  \begin{frame}
    \frametitle{Praktisches}
    Viele Menschen wollen euch Böses!
    \pause \vfill
    Traut keinen zwielichten Websites, installiert niemals (besonders unter Windows) merkwürdige Programme!
    \pause \vfill
    Informiert euch unbedingt über Skripte und Programme, bevor ihr sie ausführt!
    \pause \vfill
    Vertrauenswürdige Websites sind insbesondere \textsc{stackoverflow.com}, \textsc{superuser.com} oder \textsc{news.ycombinator.com}.
  \end{frame}

\section{Linux}

\subsection{System}

\begin{frame}
  \frametitle{Linux}
  \begin{center}
    \Huge Linux -- System
  \end{center}
\end{frame}

\begin{frame}
  \frametitle{Dateistruktur}
  ``Alles ist eine Datei'' -- Grundprinzip von Unix
  \pause \vfill
  Das Wurzelverzeichnis ist ``/'' anstelle einer Partition (``C'' unter Windows).
  \pause \vfill
  Wichige Verzeichnisse sind insbesondere:
  \begin{tabular}{lc}
    /dev & Geräte \pause  \\
    /media & Medien \pause \\
    /home & Private Dateien der Nutzer \pause \\
    /etc & Konfigurationsdateien, insb. /etc/ssl \pause \\
    /var & Variable Dateien, insb. /var/www \pause \\
    /bin & Binäre Dateien \pause \\
    /tmp & Temporäre Dateien
  \end{tabular}
  
\end{frame}

\begin{frame}
  \frametitle{Benutzerrechte}
  Dateisystem speichert Lese-/Schreib-/Nutzungsrechte für jede einzelne Datei und jeden Ordner
  \pause \vfill
  Bedeutung von Rechten bei Verzeichnissen anders.
  \pause \vfill
  Bei guter Nutzung von Rechten kann Eindringling im besten Fall nichts machen.
  \pause \vfill
  Wichtigster Nutzer: \emph{root}
  \pause \vfill
  Beispiel in Konsole.
\end{frame}

\subsection{Kommandozeile}

% cd, ls, cat, man, |, pyhton+perl+etc

\begin{frame}
  \frametitle{Kommandozeile}
  \begin{center}
    \Huge Die Kommandozeile
  \end{center}
\end{frame}

\begin{frame}
  \frametitle{Terminal, Bash und Shell}
  Eine \emph{Shell} verarbeitet Kommandozeilenbefehle und gibt eine Antwort.
  \pause \vfill
  Die \emph{Bash} ist die bekannteste Shell. Es gibt noch viele andere.
  \pause \vfill
  Ein \emph{Terminal} ist eine Art Verpackung für eine Shell, also z.B.\ das Fenster in dem die Shell läuft.
\end{frame}

\begin{frame}
  \frametitle{Wichtigste Befehle}
  \begin{tabular}{ll}
    cd & Wechsle Verzeichnis \\
    ls & Zeige Verzeichnisinhalt \\
    cat & Zeige/Gib wieder Inhalt von Textdateien an \\
    man & Zeige Hilfe zu Befehl an \\
    python/perl/gcc & Kompiliere mit entsprechender Sprache \\
    sh & Führe Shellskript aus \\
    DATEI & Führe binäre DATEI aus \\
    make & Führe make Skript aus \\
    bc -l & Taschenrechner :-)
  \end{tabular}
\end{frame}

\begin{frame}
  \frametitle{Pipes}
  Befehle in der Bash können hintereinander ausgeführt werden mittels einer Pipe ``|''. Diese gibt die Ausgabe als Eingabe an den nächsten Befehl weiter.
  \pause \vfill
  cat testdatei | uniq -u | sort
  \pause \vfill
  Gibt den Inhalt der Datei ``testdatei'' weiter an ``uniq'' mit Option ``-u'', doppelte Zeilen werden weggeschmissen und danach sortiert. 
\end{frame}

% \begin{figure}[H]
% \centering
% \includegraphics[width=0.5\linewidth]{parental_advisory.png}
% \end{figure}

% \begin{picture}(0,0)
%   \put(0,0){%
%     \includegraphics[scale=0.2]{fg1.jpg}
%    }
%    \put(6,0.5){%
%     \includegraphics[scale=0.2]{fg2.jpg}
%    }
%    \put(0,-4){%
%     \includegraphics[scale=0.2]{fg3.jpg}
%    }
%    \put(4.5,-4){%
%     \includegraphics[scale=0.2]{got_fight.jpg}
%    }
%    \put(8.5,-4){%
%     \includegraphics[scale=0.2]{book.jpg}
%    }
% \end{picture}


\end{document}
