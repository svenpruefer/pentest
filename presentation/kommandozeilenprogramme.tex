\documentclass[12pt,utf8,notheorems,compress]{beamer}

\usepackage[utf8]{inputenc}
\usepackage{default}

\usepackage{lmodern,listings}

% \usepackage{type1cm}
% \usepackage{type1ec}
% \RequirePackage{fix-cm}  %(Die Warnungen wurden schon durch lmodern beseitigt.)

% \usetheme{Berlin}
\usetheme{Warsaw}
\useoutertheme{split}
\usecolortheme{seahorse}
\usepackage{kurier}
\useinnertheme{rectangles}
\setbeamertemplate{navigation symbols}{}
\setbeamertemplate{footline}{}
\setbeamertemplate{headline}{}




% \usepackage[ngerman]{babel}
\usepackage[T1]{fontenc} % Trennung bei Woertern mit Umlauten
\usepackage{amssymb,amsmath,amsthm}
\usepackage{url,tikz}
\usepackage{graphicx}
\usepackage{comment,microtype}
\usepackage{listings}
\usepackage{pst-all}

%\usepackage{tikz-cd} % for commutative diagrams, but didn't work here
\usepackage{amsmath,amscd} % for (very simple) commutative diagrams

%\hyphenation{mani-fold}
%\hyphenation{sub-mani-fold}

\setlength\parskip{\medskipamount}
\setlength\parindent{0pt}

\newcommand{\slogan}[1]{%
  \begin{center}%
    \setlength{\fboxrule}{2pt}%
    \setlength{\fboxsep}{-3pt}%
    {\usebeamercolor[fg]{item}\fbox{\usebeamercolor[fg]{normal
    text}\parbox{0.9\textwidth}{\begin{center}#1\end{center}}}}%
  \end{center}%
}

% top color=#2!10, bottom color=#2!90

\newcommand{\mybox}[2]{%
         \begin{center}%
            \begin{tikzpicture}%
                \node[rectangle, draw=#2, rounded corners=5pt, inner xsep=5pt, inner ysep=6pt, outer ysep=10pt]{
                \begin{minipage}{0.75\linewidth}#1\end{minipage}};%
            \end{tikzpicture}%
         \end{center}%
}

\everymath{\displaystyle}

\title{Hacking Workshop -- Mathecamp 2016 in Windischleuba}

\author{Sven Prüfer}

%\institute{Institut f\"ur Mathematik, Universit\"at Augsburg}

\date{\today}

%\titlegraphic{
% \includegraphics[width=8cm]{./Uni_Aug_Logo_MNF_CMYK.pdf}
%}

\setlength{\unitlength}{1cm}

\begin{document}

\setbeameroption{show notes}
\setbeamertemplate{note page}[plain]

\begin{frame}
 \titlepage
\end{frame}

\begin{frame}[shrink=25]
%\frametitle{Inhalt}
\vspace{1cm}
\tableofcontents
\end{frame}


\section{Wichtige Kommandozeilenwerkzeuge}

\subsection{Kommunikation über Netzwerke}

\begin{frame}
  \frametitle{Curl und wget}
  
\end{frame}

\begin{frame}
  \frametitle{Netcat}
  
\end{frame}

\begin{frame}
  \frametitle{Telnet}
  
\end{frame}

\begin{frame}
  \frametitle{SSH und SCP}
  
\end{frame}

\begin{frame}
  \frametitle{tcpdump}
  
\end{frame}

\begin{frame}
  \frametitle{Wireshark}
  
\end{frame}

\subsection{Reconnaissance}

\begin{frame}
  \frametitle{Host}
  
\end{frame}

\begin{frame}
  \frametitle{Whois}
  
\end{frame}

\begin{frame}
  \frametitle{Ping}
  
\end{frame}

\begin{frame}
  \frametitle{Nmap}
  
\end{frame}

\begin{frame}
  \frametitle{Nmap Resultate}
  
\end{frame}

\begin{frame}
  \frametitle{Nmap OS Erkennung}
  
\end{frame}

\begin{frame}
  \frametitle{Traceroute}
  
\end{frame}


\end{document}
